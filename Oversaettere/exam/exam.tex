\documentclass[11pt,a4paper]{article}

\usepackage[utf8]{inputenc}
\usepackage[english]{babel}
\usepackage[T1]{fontenc}
\usepackage{listings}
\usepackage{amsmath,amssymb,amsfonts}
\usepackage{graphicx}
\usepackage{courier}
\usepackage{listings}
\usepackage{color}

\definecolor{dkgreen}{rgb}{0,0.6,0}
\definecolor{gray}{rgb}{0.5,0.5,0.5}
\definecolor{mauve}{rgb}{0.58,0,0.82}

\usepackage{float}
\newfloat{Code}{H}{myc}


\lstset{ %
  backgroundcolor=\color{white},  % choose the background color; you must add \usepackage{color} or \usepackage{xcolor}
  basicstyle=\scriptsize\ttfamily,       % the size of the fonts that are used for the code
  breakatwhitespace=false,        % sets if automatic breaks should only happen at whitespace
  breaklines=true,                % sets automatic line breaking
  captionpos=b,                   % sets the caption-position to bottom
  commentstyle=\color{dkgreen},   % comment style
  deletekeywords={...},           % if you want to delete keywords from the given language
  escapeinside={\%*}{*)},         % if you want to add LaTeX within your code
  frame=single,                   % adds a frame around the code
  keywordstyle=\color{blue},      % keyword style
  language=ml,                % the language of the code
  morekeywords={*,...},           % if you want to add more keywords to the set
  rulecolor=\color{gray},        % if not set, the frame-color may be changed on line-breaks within not-black text (e.g. comments (green here))
  showspaces=false,               % show spaces everywhere adding particular underscores; it overrides 'showstringspaces'
  showstringspaces=false,         % underline spaces within strings only
  showtabs=false,                 % show tabs within strings adding particular underscores
  stringstyle=\color{mauve},      % string literal style
  tabsize=2,                      % sets default tabsize to 2 spaces
  title=\lstname                  % show the filename of files included with \lstinputlisting; also try caption instead of title
}

\title{Report for ordinary compiler exam at DIKU - Taskset 4}
\author{Jonas Brunsgaard (msn368)}

\begin{document}
\maketitle
\newpage
\tableofcontents
\newpage
\section*{Shot commings}

\begin{itemize}
  \item When running the whole test suite with the optimizer enabled, some tests still fail
  \item The handed out test \texttt{testSplitConcat.fo} fails due to bounds checking, which in my opinion are correct 
\end{itemize}

\section{Top-Down Parsing for if-then-else}
In this assignment we will work our way forward to determine whether or not
a grammar is LL(1). This can be determined by deriving the Look-ahead sets for
the grammar.

To derive the Look-ahead sets we need to eliminate any left
recursion, left-factorize the grammar, determine nullabilility, first sets
and follow sets.

The procedure used in this assignment are described in
the book section 2.12, With the one exception that we use Look-ahead sets
instead of a parse table to determine if the grammar is LL(1). Look ahead sets
are described in the slides from the parser lecture.

\subsection{Left-factorization of the grammar}
Below is the new grammar after left-factorization.
$$
\begin{array}{lcl}
    S & \rightarrow & \mathrm{if\:}BSS_* \\
    S & \rightarrow & \mathrm{return\:NUM;} \\
    S_* & \rightarrow & \varepsilon \\
    S_* & \rightarrow & \mathrm{else}\:S \\
    B & \rightarrow & \mathrm{(\,NUM\,)} \\
\end{array}
$$

\subsection{Nullablility and First sets}
First we determine if terminals and nonterminals are \emph{nullable}.

\begin{center}
\begin{tabular}{c||c|c|c}
Right-hand side & Init & First Iter & Sec Iter\tabularnewline
\hline 
\hline 
$\mathrm{if\:}BSS*$ & \emph{false} & \emph{false} & \emph{false}\tabularnewline
\hline 
$\mathrm{return\:NUM;}$ & \emph{false} & \emph{false} & \emph{false}\tabularnewline
\hline 
$\varepsilon$ & \emph{false} & \emph{true} & \emph{true}\tabularnewline
\hline 
$\mathrm{else}\:S$ & \emph{false} & \emph{false} & \emph{false}\tabularnewline
\hline 
$\mathrm{(\,NUM\,)}$ & \emph{false} & \emph{false} & \emph{false}\tabularnewline
\hline 
Nonterminal & \multicolumn{1}{c}{} & \multicolumn{1}{c}{} & \tabularnewline
\hline 
$S$ & \emph{false} & \emph{false} & \emph{false}\tabularnewline
\hline 
$S_*$ & \emph{false} & \emph{true} & \emph{true}\tabularnewline
\hline 
$B$ & \emph{false} & \emph{false} & \emph{false}\tabularnewline
\end{tabular}
\end{center}
Then we use fixed point iteration to determine the \emph{FIRST}-sets:
\begin{center}
\begin{tabular}{c||c|c|c}
Right-hand side & Init & First Iter & Sec Iter\tabularnewline
\hline 
\hline 
$\mathrm{if\:}BSS_*$       & $\emptyset$ & \{if\}        & \{if\}        \tabularnewline
\hline                                                                
$\mathrm{return\:NUM;}$ & $\emptyset$ & \{return\}    & \{return\}    \tabularnewline
\hline                                                                
$\varepsilon$           & $\emptyset$ & $\emptyset$   & $\emptyset$   \tabularnewline
\hline                                                                
$\mathrm{else}\:S$      & $\emptyset$ & \{else\}      & \{else\}      \tabularnewline
\hline                                                                
$\mathrm{(\,NUM\,)}$    & $\emptyset$ & \{(\}         & \{(\}         \tabularnewline
\hline 
Nonterminal & \multicolumn{1}{c}{} & \multicolumn{1}{c}{} & \tabularnewline
\hline 
$S$                     & $\emptyset$ & \{if,return\} & \{if,return\} \tabularnewline
\hline                                                                  
$S_*$                   & $\emptyset$ & \{else\}      & \{else\}      \tabularnewline
\hline                                                                 
$B$                     & $\emptyset$ & \{(\}         & \{(\}         \tabularnewline
\end{tabular}               
\end{center}                

\subsection{Calculate Follow sets for all nonterminals}
By following the procedure found on page 59 in the book, we derive the table
below. To handle the end-of-string condition we add $S'\rightarrow S\$$
to the production.

\begin{center}
    \begin{tabular}{ll}
\hline 
Production & Constraints\tabularnewline
\hline 
$S'     \rightarrow S\$$                    & $\{\$\}\subseteq FOLLOW(S)$ \tabularnewline
$S      \rightarrow \mathrm{if\:}BSS_*$     & \begin{minipage}[t]{0.5\columnwidth}
                                                $\{\mathrm{return, if}\}\subseteq FOLLOW(B)$,\\
                                                $FOLLOW(S) \subseteq FOLLOW(S_*)$,\\
                                                $\{\mathrm{else}\} \subseteq FOLLOW(S)$
                                                \end{minipage}
                                                \tabularnewline

$S      \rightarrow \mathrm{return\:NUM;}$  & \tabularnewline
$S_{*}  \rightarrow \varepsilon$            & \tabularnewline
$S_{*}  \rightarrow \mathrm{else}\:S$       & $FOLLOW(S_*) \subseteq FOLLOW(S)$ \tabularnewline
$B      \rightarrow \mathrm{(\:NUM\:)}$     & \tabularnewline
\hline
\end{tabular}
\end{center}
We first use the constraints $ \{\$\} \subseteq FOLLOW(S)$ and constraints
of the form $FIRST(\dots)\subseteq FOLLOW(\dots)$ to get the initial sets.
\begin{eqnarray*}
    FOLLOW(S)       & \subseteq & \{\mathrm{else, \$} \} \\
    FOLLOW(S_*)     & \subseteq & \{\emptyset\} \\
    FOLLOW(B)       & \subseteq & \{\mathrm{if, return}\} \\
\end{eqnarray*}
and then use the constrains on the form $FOLLOW(\dots)\subseteq FOLLOW(\dots)$:
\begin{eqnarray*}
    FOLLOW(S)       & \subseteq & \{\mathrm{else, \$} \} \\
    FOLLOW(S_*)     & \subseteq & \{\mathrm{else, \$} \} \\
    FOLLOW(B)       & \subseteq & \{\mathrm{if, return}\} \\
\end{eqnarray*}
\subsection{Look-aheads sets}
From the lecture slides the look ahead set is defined as
$$la(X\rightarrow\alpha)=
\begin{cases}
    FIRST(\alpha)\cup FOLLOW(X) & \textrm{, if }NULLABLE(\alpha)\\
    FIRST(\alpha)               & \textrm{, otherwise}
\end{cases}$$
Below you see the lookahead sets for our productions.
$$
\begin{array}{lcl}
    LA(S   \rightarrow  \mathrm{if\:}BSS_*)     & = &      \{\mathrm{if}\} \\
    LA(S   \rightarrow  \mathrm{return\:NUM;})  & = &      \{\mathrm{return}\} \\
    LA(S_* \rightarrow  \varepsilon)            & = &      \{\mathrm{else,} \$\} \\
    LA(S_* \rightarrow  \mathrm{else}\:S)       & = &      \{\mathrm{else}\} \\
    LA(B   \rightarrow  \mathrm{(\,NUM\,)})     & = &      \{\mathrm{(}\} \\
\end{array}
$$
From the course slides on parsing we know that if for each nonterminal $X
\in N$ in grammar $G$, all productions of $X$ have disjoint look-ahead sets,
then the grammar $G$ is LL(1). This is not the case in out grammar d to the
production $S_*$, an thus our production is not LL(1)
\subsection{Resolving the ambiguity in a recursive-decent parser}
As mentioned in the assignment the established convention when dealing with
the dangling else is to attach the ``else'' to the nearby ``if'' statement.

In our case - as in the book - we can achieve this by removing the empty
production of $S_*$. In more general we have to try rewriting the ambiguous
grammar to a unambiguous one.

Normally this is done by adding productions to keep track of the
statement state(matched/unmatched). An example can be found in the book page 52 or at -
http://www.parsifalsoft.com/ifelse.html.

Also ambiguity can be resolved adding more terminals to the grammar, e.g the
terminal "fi" - as used in Bash - which indicates the end of the conditionally
executed code, or we could add the more common used braces to the grammar.

\section{Implementing Band (bitwise and)}

\subsection{Code changes to implement \texttt{band}}

In this sub-assignment the bitwise \texttt{AND} operator \texttt{band} has
been implemented. The first changes were made to the structure file -
\texttt{Fasto.sml} - where I added the AbSyn node as part of the datatype
\texttt{Exp} and implemented pretty printing functionality for the new node.

Afterwords \texttt{band} was added to the lexer by adding the keyword
\texttt{band} and thereafter the parser, by adding the token \texttt{BAND} and
providing it with left association and the correct precedence, as shown below.
\begin{lstlisting}
...
%left PLUS MINUS NEGATE
%left BAND
%left TIMES DIV
...
\end{lstlisting}
Also in the parser \texttt{BAND} was added to the $Exp$ production as an
independent expression(creating a \texttt{Fasto.Band} node), as well as part
of any SOAC expressions taking an operator as an argument.

Then \texttt{Fasto.Band} was added to the typechecker with the following code:
\begin{lstlisting}
| Fasto.Band (e1, e2, pos) =>
   let
     val (ts, es) = ListPair.unzip (List.map (expType vs) [e1,e2])
     (* ts holds the type, we unify with int *)
     val t  = List.foldl (unifyTypes pos)(Fasto.Int pos) ts
   in
     (Fasto.Int pos, Fasto.Band (List.nth (es,0),
                                 List.nth (es,1), pos))
   end
\end{lstlisting}
The code works by creating the list \texttt{ts} holding the
types of the of the two expressions applied to \texttt{band}. The types in
\texttt{ts} is unified with \texttt{Int}. The function \texttt{unifyTypes}
will raise an error exception if the types are not \texttt{Int}. If no error
is raised a tuple is returned containing the type of \texttt{band} and the
\texttt{band} expression.

Finally I added the case for \texttt{Fasto.Band} to the compiler:
\begin{lstlisting}
| Fasto.Band (e1,e2,pos)=>
    let val t1 = "_band1_"^newName()
        val t2 = "_band2_"^newName()
        val code1 = compileExp e1 vtable t1
        val code2 = compileExp e2 vtable t2
    in  code1 @ code2 @
        [Mips.AND (place,t1,t2)]
    end
\end{lstlisting}
The code above recursively (by calling compileExp) determines the
values of the input expressions and place them in registers \texttt{t1} and
\texttt{t2}. \texttt{Mips.AND} then perform the bitwise operation on the two
registers and place the result in \texttt{place}. Note that I have used only
\texttt{Mips.AND}. One would be able to check, if one of the expressions were a
\texttt{Fasto.Num} and then use \texttt{Mips.ANDI} and thus save a register.
I did not do that to keep the code in line with the existing code base (see the
\texttt{compileExp} case for for \texttt{Fasto.Plus, Fasto.Minus} etc..

Finally cases for \texttt{Fasto.Band} were implemented in
\texttt{Optimization.sml} for all existing functions, ass well as the
functions for for constant propagation and constant folding. Below the node
for constant folding are shown. I used the MosML word-library to achieve this
functionality:
\begin{lstlisting}
 | Fasto.Band(e1,e2,p) =>
     let
       val (s1, e1new) = ctFoldP( e1, vtab )
       val (s2, e2new) = ctFoldP( e2, vtab )
       fun bitwiseand (n1,n2) =
         let
           val w1 = Word.fromInt(n1)
           val w2 = Word.fromInt(n2)
         in
           Word.toInt(Word.andb(w1,w2))
         end
     in
       if
         (isCtVal e1new) andalso (isCtVal e2new)
       then
         (true,evalBinop(bitwiseand, e1new, e2new, p))
       else
         case (e1new, e2new) of
             ( Num(0,_), _         ) => (true,Fasto.Num(0, p))
           | ( _       , Num(0,_ ) ) => (true,Fasto.Num(0, p))
           | ( _       , _         ) => (s1 orelse s2,
                                         Fasto.Band(e1new,e2new,p)
\end{lstlisting}
\subsection{Correctness of Band}
To ensure correctness of \texttt{band} the follow tests has been added to the
\texttt{DATA} folder. The content of the .out files have been calculated by
hand.
\begin{description}
  \item[testBand.fo] \hfill \\
  Testing the correctness of \texttt{band} when applied to integers.
  \item[testBandprecedens.fo] \hfill \\
  Testing the correctness of \texttt{band} in regard to precedence and
  associativity.
  \item[testBandArrayError.fo] \hfill \\
  Testing the correctness of \texttt{band} when used with \texttt{op} as
  function argument..
  \item[testBandArrayError.fo] \hfill \\
  Type testing with an Array as one of the arguments.
  \item[testBandBoolBothOperands.fo] \hfill \\
  Type testing with booleans as both arguments.
  \item[testBandBoolFstOperandErr.fo] \hfill \\
  Type testing with boolean as first argument.
  \item[testBandBoolSecondOperandErr.fo] \hfill \\
  Type testing with boolean as second argument.
  \item[testBandCharBothOperands.fo] \hfill \\
  Type testing with chars as both arguments.
  \item[testBandCharFstOperandErr.fo] \hfill \\
  Type testing with char as first argument.
  \item[testBandCharSecondOperandErr.fo] \hfill \\
  Type testing with char as second argument.
\end{description}

\subsection{Source code}
The full source code can be found in the \texttt{SRC} folder and where changes have
been made there should be a comment before and after the code block. I put my
exam number(msn378) in this comment so it would be easy to find by searching the files.

\section{Implementation of the second order array combinators \texttt{split}
and \texttt{concat}}
In this sub assignment task was to implement the second-order-array-combinators \texttt{split}
and \texttt{concat} into the Fasto language and the compiler.

\subsection{Code changes}
I started by adding the AbSyn nodes for \texttt{split} and \texttt{concat}
to the structure in \texttt{Fasto.sml} as described in the handed out exam
assignment. Also pretty printing functionality were implemented here.

\subsubsection{Lexer}
As specified in the exam paper, the tokens \texttt{split} and \texttt{concat}
were added to the list of keywords in the lexer with the following code:
\begin{lstlisting}
| "split"        => Parser.SPLIT pos
| "concat"       => Parser.CONCAT pos
\end{lstlisting}

\subsubsection{Parser}
In the parser \texttt{split} and \texttt{concat} were added to the right hand
side of the $Exp$ production.
\begin{lstlisting}
 | SPLIT LPAR Exp COMMA Exp RPAR
                 { Fasto.Split ($3, $5, Fasto.UNKNOWN, $1) }
 | CONCAT LPAR Exp COMMA Exp RPAR
                 { Fasto.Concat ($3, $5, Fasto.UNKNOWN, $1) }
\end{lstlisting}
Note, at this point we do not know the type of the elements in the input
array. Therefore we set the type in the to \texttt{Fasto.UNKNOWN}.
The type is determined in the type checking phase.

\subsubsection{Type checker}

\paragraph{For \texttt{split}} the following case was added to the type checker. The full
code sample is added due to the it informative comments.
\begin{lstlisting}
 | Fasto.Split (num, arr, typ, pos)
   => let
       (* type of first argument, should be Int *)
       val (n_type, n_dec) = expType vs num
       (* type of second argument, should be Array *)
       val (arr_type, arr_dec) = expType vs arr
       (* type of elements in Array, if not Array an 
        * exception is thrown *)
       val el_type = case arr_type of
             (* typ is Unknown, we unify with the types and
             * determine the elements types of the Array..
             * Unifying are not stricly necessary here, but
             * we do it to keep consistency in the code eg.
             * Reduce also uses this practice,
             * alternativly we could just return t
             * the first tuple element in Array *)
             Fasto.Array (t,_) => unifyTypes pos (typ, t)
           | _ => raise Error ("Split: Second argument"^
                               "not an array",pos)
      in
        (* check if first argument is type Int *)
        if
          typesEqual (n_type, Fasto.Int pos)
        then
          (Fasto.Array (arr_type, pos),
           Fasto.Split (num ,arr_dec, el_type, pos))
        else
          raise Error ("Split: First argument type is not Int " ^
                       showType n_type, pos)
      end
\end{lstlisting}
With the comments hopefully it is easy to determine how the case is handled
by the type checker.

For the type checking to be successful the first expression must have
the type \texttt{Fasto.Int}, the second expression must be of type
\texttt{Fasto.Array}. The type of the elements in the array is not
important to whether the type checking succeed. The elements could actually be
\texttt{Fasto.UNKNOWN} and still pass type checking. This behavior may not be
intentional and could be fixed by a different ordering of the cases in
the function \texttt{unifyTypes}, which at this point is:
\begin{lstlisting}
fun unifyTypes pos (Fasto.UNKNOWN, t) =
  | unifyTypes pos (t, Fasto.UNKNOWN) =
  | unifyTypes pos (t1, t2) = (* typesEqual where
                                 unknown and unknown
                                 return false. but
                                 unifyTypes has already
                                 returned due to case
                                 ordering *)
\end{lstlisting}

\paragraph{For \texttt{concat}} the following case was added to the type checker. The full
code sample is again added below due to the comments explaining how it
works.
\begin{Code}
\begin{lstlisting}
| Fasto.Concat (arr1, arr2, arg1_typ, pos)
  => let
      (* type of first argument, should be Array *)
      val (arr1_type, arr1_dec) = expType vs arr1
      (* type of second argument, should be Array *)
      val (arr2_type, arr2_dec) = expType vs arr2
      (* type of elements in Array, if not Array and exception is thrown *)
      val el1_type = case arr1_type of
            Fasto.Array (t,_) => unifyTypes pos (arg1_typ, t)
          | other => raise Error ("Concat: First Array Argument "^
                                  "not an array",pos)
      val el2_type = case arr2_type of
            Fasto.Array (t,_) => unifyTypes pos (arg1_typ, t)
          | other => raise Error ("Concat: Second Array Argument "^
                                  "not an array",pos)
      val el_type_unified = unifyTypes pos (el2_type, el1_type)
     (* An alternative solution would be to check with typesEqual, this
      * way we could have used a if-then-else and raised a more specific
      * error. *)
     in
       (Fasto.Array (el_type_unified, pos), Fasto.Concat(arr1_dec,
       arr2_dec, el_type_unified, pos))
     end
\end{lstlisting}
\end{Code}
First we find the types with the function \texttt{expType} of the two input
expression (we hope these are arrays, therefore the naming \texttt{arr1} and
\texttt{arr2} of the input parameters).

Then we pattern match on the types of the expressions \texttt{arr1} and
\texttt{arr2}. If the pattern do not match the type \texttt{Fasto.Array} we
raise the exception Error and inform the user with the error message.

Now we have the element types of the two arrays, thus we are able to unify
these and thus check equality.

If no exception if thrown in the call to \texttt{unifyTypes}, then we can
safely proceed and return a tuple holding the type of the \texttt{concat}
expression and a updated \texttt{concat} expression.

\subsection{Compiler}
\paragraph{For \texttt{split}} a case was added to compileExp.
Below are shown some pseudo code that roughly describes the idea of the implementation.
\begin{lstlisting}
split(int n, int[] lst, type eltp);
{
int *data1_adr;
int *data2_adr;

// Check bounds
if (n-1 >= 0){ THROW( E)} 
if (n-len(lst) >= 0){ THROW( E)} 

// Get adresses where data is already stored
data1_adr = lst[0];
data2_adr = sizeof(elty) * n + data1_adr 

// Create tow new inner array (only with header information)
int arr1[1]
int arr2[1]

// Create the outer array
int arr2[1]

// Set number of elements
arr1[0] = n
arr2[0] = data2_adr-data1_adr

// Copy over adresses for existing data
arr1[1] = data1_adr
arr2[1] = data2_adr

// The outer Array has two elements, arr1 and arr2.
arr3[0] = 2
arr3[1] = &[arr1,arr2] 
}
\end{lstlisting}
By this implementation we avoid allocation of memory for data already placed
in memory.
And thus, we only need to allocate new ``data entries'' in the memory for the
outer array holding the two new inner arrays, but due to the way arrays work
in Fasto, we still have to allocate two words of memory for every new array to
hold the header information.

In theory we could avoid allocation any new memory by placing the data in the
registers. But it seams like a bad solution because registers are spare and
a single call to the \texttt{split} method, would take $2*2 = 4$ registers to
hold the information about the new inner arrays. And $2+2 = 4$ registers to
hold information and data
of the outer array. 

Another solution to avoid memory allocation, would present itself if the array
to be split was not used anywhere else in the program. Then we could be smart
and ``reuse'' the array and thus saving the allocation of 2 words for the
header information. But it would be essential for this to work, that the array
we reuse 'dead'

\paragraph{For \texttt{concat}} I also added a new case. \texttt{concat} first
checks, whether the two lists to be joined are adjacent. This is done by Mips
code with the same approach as in the pseudo code below
\begin{lstlisting}
// data_addr holds the memory address's where data start for the
// two arrays to be joined. If adj is 0, the two arrays can be joined
// without copy over data

int adj
adj = (data_addr1 - data_addr2) - (len(lst1) * sizeof(lst1_eltp))
\end{lstlisting}
If the arrays are adjacent we use \texttt{dynalloc} to create an 'empty'
array only containing header information(thus avoiding memory allocation).

If the two lists are not adjacent, then a new array are created with the size
len(arr1) + len(arr2). And the data is copied over.
\begin{lstlisting}
// n is size of final array
int n
n = len(arr1) + len(arr2)

// pointers to exsiting data
eltp* data_addr1;
eltp* data_addr2;

//final array
eltp final[n]

// Loop n times to copy all elements to new array
for (int i=0; i >= n; i++)
{
// if i is less than len(arr1) copy 
// data from arr1 else copy data from arr2

if (i < len(arr1)
{
final[i] = *data_addr1
data_addr1 = data_addr1 + sizeof(eltp) 
}
final[i] = *data_addr2
data_addr2 = data_addr2 + sizeof(eltp) 
}
\end{lstlisting}

\subsubsection{Optimizer}
Cases for \texttt{concat} and \texttt{split} has been added to the existing functions in the optimizer as
well, as to the functions performing constant folding and constant propagation.

\subsection{Memory allocation in the example}
The \texttt{split}ting will make an outer array of size 2 holding the inner
arrays, thus 2 words for data plus 2 words for the header information.

The data itself will be reused, but the two inner arrays has to each allocate
2 words for header information, thus at this point all in all 8 words has been allocated.

The \texttt{concat}enation will reuse the data(because it is adjacent), but
also allocate 2 words for header information. Thus, 10 words - also known
as 40 bytes -  will be allocated.

\subsection{Correctness of the code}
To ensure correctness of \texttt{concat} and \texttt{split} the following
tests has been added to the \texttt{DATA} folder.

\subsubsection{Tests for \texttt{concat}}

\begin{description}
  \item[testConcatIntCharBool.fo] \hfill \\
  Testing the correctness of \texttt{concat} when applied to arrays of bool,
  char and int.
  \item[testConcatArrays.fo] \hfill \\
  Testing the correctness of \texttt{concat} when applied to two dimentional
  array.
  \item[testConcatFstArgBoolErr.fo] \hfill \\
  Type testing with boolean as first argument..
  \item[testConcatFstArgCharErr.fo] \hfill \\
  Type testing with char as first argument..
  \item[testConcatFstArgIntErr.fo] \hfill \\
  Type testing with int as first argument..
  \item[testConcatSecndArgBoolErr.fo] \hfill \\
  Type testing with boolean as second argument..
  \item[testConcatSecndArgCharErr.fo] \hfill \\
  Type testing with char as second argument..
  \item[testConcatSecndArgIntErr.fo] \hfill \\
  Type testing with int as second argument..
\end{description}

\subsubsection{Tests for \texttt{split}}

\begin{description}
  \item[testSplitIntBoolChar.fo] \hfill \\
  Testing the correctness of \texttt{split} when applied to arrays of boolean,
  char and int.
  \item[testSplitTwoDimArray.fo] \hfill \\
  Testing the correctness of \texttt{concat} when applied to two dimensional
  array.
  \item[testSplitLowerBound.fo] \hfill \\
  Testing correctness of lower bound handling.
  \item[testSplitUpperBound.fo] \hfill \\
  Testing correctness of upper bound handling.
  \item[testSplitFstArgumentArrayErr.fo] \hfill \\
  Type testing with array as first argument.
  \item[testSplitFstArgumentBoolErr.fo] \hfill \\
  Type testing with boolean as first argument.
  \item[testSplitFstArgumentCharErr.fo] \hfill \\
  Type testing with char as first argument.
  \item[testSplitSecndArgumntBoolErr.fo] \hfill \\
  Type testing with boolean as second argument.
  \item[testSplitSecndArgumntChar.fo] \hfill \\
  Type testing with char as second argument.
  \item[testSplitSecndArgumntInt.fo] \hfill \\
  Type testing with int as second argument.
\end{description}

\subsection{Source code}
The full source code for \texttt{split} and \texttt{concat} can be found in
the \texttt{SRC}. Search for 'msn378' to easily find the code blocks I have
added or modified.

\section{Implementing Constant Folding} 
In this sub assignment constant folding have been implemented into
\texttt{Optimization.sml}. The idea of constant folding is simple, if you have
an expression $e$ holding other expression of constants, fold $e$ to a simpler
expression if possible.
In the optimizer all expression are uniquely named as a variable and put into
a \texttt{Fasto.Let} expression. Thus, constant folding does nothing unless
also constant propagation is used. Please note, that I have also completed the
function performing constant propagation. 

The two optimization methods go hand in hand, in the way that a constant
propagation run through, may lead to new possible optimizations with constant
folding, and visa versa.

A lot of new code has been implemented, but instead of showing the rather
uniform code here, I will instead summarize the cases where I have found to be able
to make optimizations, and let you have a look at the code yourself. Note that
much of the code implemented in the optimized relies on proper type
checking, so when er get a expression of type \texttt{Fasto.Plus}, we assume
that the types of the sub-expressions are of \texttt{Fasto.Int}.


\subsection{Cases of optimization}
Below is mentioned the folding optimizations I have done.

\begin{description}
    \item[\texttt{Fasto.Plus}] \hfill \\
      If both input expressions are a \texttt{Num}, a \texttt{Fasto.Num} is returned.
      If one of the expression is a \texttt{Num} with the value 0, the other expression is returned.
  \item[Fasto.Minus] \hfill \\
      If both expressions are a \texttt{Num},  a \texttt{Fasto.Num} is returned.
      If $e1$ is a \texttt{Num} with the value 0, then Negate($e2$) is returned.
      If $e2$ is a number with the value 0, then $e2$ is returned.
  \item[Fasto.Less] \hfill \\
      If $e1$ and $e2$ are constants they are fold with op<, and a \texttt{Log} expression is returned.
  \item[Fasto.If] \hfill \\
      If $e1$ (the condition) is a constant, $e2$ is returned if $e1$ is true, otherwise $e3$ is returned as $e1$ is false.
  \item[Fasto.Times] \hfill \\
      If $e1$ and $e2$ are both constants, they are multiplied and a \texttt{Num} expression is returned.
      If $e2$ is a number-constant with the value 0 a \texttt{Num} expression with value 0 is return.
      If $e1$ is a number-constant with the value 0 a \texttt{Num} expression with value 0 is return.
      If $e1$ is a number-constant with the value 1, $e2$ is returned.
      If $e2$ is a number-constant with the value 1, $e2$ is returned.
  \item[Fasto.Divide] \hfill \\
      If $e1$ and $e2$ are both number-constants, they are divided and a \texttt{Num} expression is returned.
      If $e1$ is a number-constant with the value 0 a \texttt{Num} with value 0 is return.
      If $e2$ is a number-constant with the value 0 a exception is thrown.(you cannot divide with zero)
      If $e2$ is a number-constant with the value 1, $e1$ is returned
  \item[Fasto.Band] \hfill \\
      If $e1$ and $e2$ are constants, they are evaluated and a \texttt{Num} expression is returned.
      If either $e1$ or $e2$ is a number-constant with the value 0, a \texttt{Num} with value 0 is returned.
  \item[Fasto.And] \hfill \\
      If $e1$ and $e2$ are both constants with the value true, a \texttt{Log} expression with value \texttt{true} is returned.
      If either $e1$ or $e2$ are false, a \texttt{Log} expression is returned with the value \texttt{false}. 
      If $e1$ is \texttt{true}, $e2$ is returned
      If $e2$ is \texttt{true}, $e1$ is returned
  \item[Fasto.Or] \hfill \\
      If either $e1$ or $e2$ is a boolean-constant with the value \texttt{true}, a \texttt{Log} expression with value \texttt{true} is returned.
      if $e1$ is a boolean-constant with the value \texttt{false}, $e2$ is returned
      if $e2$ is a boolean-constant with the value \texttt{false}, $e1$ is returned
  \item[Fasto.Not] \hfill \\
      if $e1$ is a boolean-constant with the value \texttt{false}, it is folded to a \texttt{Log} with value \texttt{true}
      if $e1$ is a boolean-constant with the value \texttt{true}, it is folded to a \texttt{Log} with value \texttt{false}
  \item[Fasto.Negate] \hfill \\
      if $e1$ is a number-constant, a \texttt{Num} expression with the negated constant is returned.
      
\end{description}

\subsection{Correctness of Constant folding}
The way I have tested constant folding was by using the already exsiting
test files in \texttt{DATA}. By implementing constant propagation I was was
able to see the code folding when compiled with the FastoC -o, and by removing
the print statements from the optimizer and adding \texttt{-o} as argument in
\texttt{testg.sh}, I was able to run all the test with optimization enabled and check
correctness.

\subsection{Source code}
The full source code can be found in the \texttt{SRC}, in the file \texttt{Optimization.sml}
\end{document}
