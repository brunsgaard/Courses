\documentclass[11pt,a4paper]{article}

\usepackage[utf8]{inputenc}
\usepackage[english]{babel}
\usepackage[T1]{fontenc}
\usepackage{listings}
\usepackage{amsmath,amssymb,amsfonts}
\usepackage{graphicx}

\title{Individual Compiler Assignment - w3}
\author{Jonas Brunsgaard}

\begin{document}
\maketitle

\section{Exercise 1}
\subsection{1.a}
In Call-by-value, anything passed into a function call is unchanged in the
caller's scope when the function returns.
\begin{lstlisting}
f(5, 2, 5){
    a = b + c           :a = 7
    b = c - b           :b = 3
    c = c + a           :c = 12
    return(7 + 3 + 12)  :+(a,b,c) = 22
}
\end{lstlisting}
thus the function will print 22,5,2.

\subsection{1.b}
In Call-reference, a function receives an implicit reference to a variable
used as argument, rather than a copy of its value.

\begin{lstlisting}
f(5, 2, 5){
    a = b + c           :a = c = 7
    b = c - b           :b = 7 - 2 = 5
    c = c + a           :c = a = 7 + 7 = 14
    return(14 + 5 + 14)  :+(a,b,c) = 33
}
\end{lstlisting}
thus the function will print 33,14,5.

\subsection{1.c}
In Call-value-result, is a special case of call-by-reference where the
provided reference is unique to the caller. In this exercise the copies will
be made in left-to-right order both on entry and on return.

\begin{lstlisting}
f(5, 2, 5){
    a = b + c           :a = 5 + 2 = 7
    b = c - b           :b = 5 - 2 = 3
    c = c + a           :c = 5 + 7 = 12
    return(7 + 3 + 12)  :+(a,b,c) = 22
}
\end{lstlisting}
we copy back left to right, thus x = a
$$
\begin{array}{lcl}
    x & = & a \\
    y & = & b \\
    x & = & c
\end{array}
$$
thus the function will print 22,12,3.
\section{Exercise 2}
\subsection{2.a - static scoping}
The call $f(4)$ will print $5$ \\\\
The call $f(7)$ will print $5$

\subsection{2.b - dynamic scoping}
The call $f(4)$ will print $4$ \\\\
The call $f(7)$ will print $9$

\subsection{2.c}
It could be implemented as a simple persistent symbol tables, as described in
the book page 93. The \emph{enter} and \emph{exit} functionality does
not need to be implemented. An empty list is used for the symbol table. When
binding a (key, value) pair are added to front of the list. When looking up an
identifier just iterate through the list. Return the value of the first
matching key/value pair, if no pair if found throw an 'unbound' exception.

\section{Exercise 3}
\subsection{3.a}
\begin{tabular}{l l l}
    Exp                 & Type                                          & Unify\\
    f                   & b1 * b2 * b3 -> c                             & \\
    x                   & b1                                            & \\
    y                   & b2                                            & \\
    z                   & b3                                            & \\ 
                        &                                               & \\ \hline
    map                 & (a1 -> a2) * [a1] -> [a2]                     & \\
    length              & [a3]-> int                                    & \\
    map(length,x)       & [int]                                         & a1 = [a3] \\
                        &                                               & a2 = int \\
                        &                                               & b1 = [[a3]] = [a1] \\
    (map(length,x),x)   & [int] * [[a3]]                                & \\
                        &                                               & \\ \hline
    length              & [a4] -> int                                   & \\
    length(x)           & int                                           & \\
    =                   & a5 * a5 -> bool                               & \\
    2                   &                                               & \\ 
    =(length(x),2)      & bool                                          & a5 = int \\ 
                        &                                               & \\ \hline
    map                 & (a6 -> )a7*[a6]-> [a7]                        & \\
    redint              & [int] -> int                                  & \\
    map(redint,y)       & [int]                                         & a6 = [int]   \\
                        &                                               & a7 = int \\
                        &                                               & b2 = [[int]] \\
    (map(redint,y),z)   & [int]*  a8                                    & a8 = b3\\
                        &                                               & \\ \hline
    IF                  & bool * a9 * a9 -> a9                          & \\ 
    IF(...)             & [int]*[[a3]]                                  & a8 = [[a3]] \\
                        &                                               & c = [int] * [[a3]]\\
                        &                                               & \\ \hline
    x                   & [[a3]]                                        & \\
    y                   & [[int]]                                       & \\
    z                   & [[a3]]                                        & \\
    f(x,y,z)            & [int]*[[a3]]                                  & \\
    f                   & [[a3]]*[[int]]*[[a3]] -> [int]*[[a3]]         &
\end{tabular}
\subsection{3.b}
JPEG image of the solution uploaded to Absalon.
\end{document}
