\documentclass[11pt,a4paper]{article}

\usepackage[utf8]{inputenc}
\usepackage[english]{babel}
\usepackage[T1]{fontenc}
\usepackage{clrscode3e}
\usepackage{multicol,lipsum}

\usepackage{amsmath,amssymb,amsfonts}

\title{CLRS, 2-2 Corectness of bubblesort}
\author{Jonas Brunsgaard}

\begin{document}
\maketitle

\begin{itemize}
    \item[a.] A' has to be a permutation of A.
    \item[b.] \textit{Loop invariant (2-4:)} In the subarray  $A[j...n]$, $j$ are the minimum element in the array. Also $A'[j...n]$ is a permutation of $A[j...n]$.
        \begin{itemize}
            \item[\textit{i.}] \textit{Initialization} At this point $j=n$ and therefore $A[j...n]=A[j]$, which satisfy the loop invariant.
            \item[\textit{ii.}] \textit{Maintenance} By the initialization in the loop invariant we know that $A[j]$ is a minimum element. 
                For each iteration we see that if $A[j] < A[j-1]$, then $A[j]$ and $A[j-1]$ are swapped, thus $A[j-1]$ will be a minimum element for the next
                iteration where it is $A[j]$. The loop invariant then holds for maintenance.
            \item[\textit{iii.}] \textit{Termination} The loop will terminate when $j=i$, at this point $A[i]$ will be the minimum element in $A[i...n]$.
        \end{itemize}
    \item[c.] \textit{Loop invariant (1-4):} The subarray $A[1...i-1]$ appended with $A[i...n]$ will be a permutation of the original $A$. The subarray array 
        $A[1...i-1]$ will be sorted, which follows from \textit{b.}. Also all the elements in $A[1...i-1]$ will be smaller or equal to the the elements in $A[i...n]$.
        \begin{itemize}
            \item[i.]
        \end{itemize}
    \item[d.]
\end{itemize}

\end{document}

