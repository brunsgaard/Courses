\documentclass[english]{article}
\usepackage[T1]{fontenc}
\usepackage[latin9]{inputenc}
\usepackage{geometry}
%\geometry{verbose,tmargin=2.5cm,bmargin=2.5cm,lmargin=2.5cm,rmargin=2.5cm}
\usepackage{setspace}
\setstretch{1.15}
\usepackage{babel}
\usepackage{amsmath}

\begin{document}
\section*{Assignment 2.1}
\subsection*{Subassignment 2.1.a}

Both partial derivatives are found using maple. The maple input/output can be found at the end of the document.

\begin{equation}
\frac{\partial Q}{\partial p}(p,q)=\sum\limits_{i=1}^n 2(pt+q-y)t
\end{equation}

\begin{equation}
\frac{\partial Q}{\partial q}(p,q)=\sum\limits_{i=1}^n (2pt+2q-2y)
\end{equation}
\subsection*{Subassignment 2.1.b}
In this subassignment i solve the partial derivatives, such that $\sum\limits_{i=1}^n t_i y_i$ and $\sum\limits_{i=1}^n y_i$ are separated on the right side of the equation.  

\begin{eqnarray}
	\frac{\partial Q}{\partial p}(p,q) & = & 0\\
	\sum\limits_{i=1}^n 2(t_i p +q-y_i)t_i & = & 0 \\
	\sum\limits_{i=1}^n 2(t_i p  +q-y)t_i & = & 0 \\
	\sum\limits_{i=1}^n 2t_i^2 p +\sum\limits_{i=1}^n 2t_i q \sum\limits_{i=1}^n 2y_i t_i & = & 0 \\
	\sum\limits_{i=1}^n 2t_i ^2 p +\sum\limits_{i=1}^n 2t_i q & = & \sum\limits_{i=1}^n 2y_i t_i\\
	\sum\limits_{i=1}^n t_i^2 p +\sum\limits_{i=1}^n t_i q & = & \sum\limits_{i=1}^n y_i t_i 
\end{eqnarray}
And the second equation:
\begin{eqnarray}
	\frac{\partial Q}{\partial q}(p,q) & = & 0 \\
	\sum\limits_{i=1}^n (2pt_i+2q-2y_i) & = & 0 \\
	\sum\limits_{i=1}^n 2pt_i+ \sum\limits_{i=1}^n 2q- \sum\limits_{i=1}^n 2y_i & = & 0 \\
	\sum\limits_{i=1}^n 2pt_i+ 2qn & = & \sum\limits_{i=1}^n 2y_i \\
	\sum\limits_{i=1}^n t_i p + nq & = & \sum\limits_{i=1}^n y_i
\end{eqnarray}
Note the step equation 11 to 12. Because $\sum\limits_{i=1}^n 2q$ is not affected by the specific $(y_i,t_i)$, it is possible to rewrite  $\sum\limits_{i=1}^n 2q =2qn$. For each dataentry we just want to add a $q$, Thus $nq$.


\subsection*{Sub assignment 2.1.c}
The equationsystem as found in subassignment 2.1.c
\begin{equation}
\begin{array}{ccc}
	\sum\limits_{i=1}^n t_i^2 p +\sum\limits_{i=1}^n t_i q & = & \sum\limits_{i=1}^n y_i t_i \\
	\sum\limits_{i=1}^n t_i p + nq & = & \sum\limits_{i=1}^n y_i
\end{array}
\end{equation}
can also be represented by a matrix. The matrix is shown below.
\begin{equation}
 \begin{pmatrix}
  \sum\limits_{i=1}^n t_i^2 & \sum\limits_{i=1}^n t_i \\
  \sum\limits_{i=1}^n t_i & n
 \end{pmatrix}
  \begin{pmatrix}
p \\ q
 \end{pmatrix}
 =
\begin{pmatrix}
 \sum\limits_{i=1}^n y_i t_i  \\ \sum\limits_{i=1}^n y_i
 \end{pmatrix}
\end{equation}

\subsection*{Subassignment 2.1.d}
Both double partial derivatives are found using maple. The maple input/output can be found at the end of the document.
\begin{equation}
	\frac{\partial^2 Q}{\partial p^2}(p,q) = \sum \limits_{i=1}^n 2t_i^2
\end{equation}

\begin{equation}
	\frac{\partial^2 Q}{\partial q^2}(p,q)= 2n
\end{equation}

\begin{equation}
	\frac{\partial^2 Q}{\partial p \partial p}(p,q)= \frac{\partial^2 Q}{\partial q \partial p}(p,q)= \sum \limits_{i=1}^n 2t_i
\end{equation}
$\frac{\partial^2 Q}{\partial p^2}(p,q)  = \sum \limits_{i=1}^n 2t_i^2$ will always be greater than 0, as long as $t \neq 0$. This is because $t_i$ is raised to the second power and thus the whole expression will always be positive.
The last thing I will prove, is that whenever $ab-c^2 > 0$ then it follows that also $D>0$. Lets have a look at $D$ and $ac-b^2$ from theorem 1 in the assignment text.\\\\
From the MinMax.pdf note we know that
\begin{eqnarray}
D &=& \frac{\partial^2 Q}{\partial p^2}(p,q) \frac{\partial^2 Q}{\partial q^2}(p,q) - (\frac{\partial^2 Q}{\partial p \partial p}(p,q))^2 \\
D &=& (\sum \limits_{i=1}^n 2t_i^2) 2n-(\sum \limits_{i=1}^n 2t_i)^2\\
\end{eqnarray}
From theorem 1, we know
\begin{eqnarray}
	ac-b^2 &=&  (\sum \limits_{i=1}^n t_i^2) n-(\sum \limits_{i=1}^n t_i)^2
\end{eqnarray}
If we take a closer look at equation 21 and equation 22 we will see that the two equations look much alike. Actually we can write $D$ as
\begin{eqnarray}
D &=& 2a2c-(2b)^2 \\
D &=& 4ac-(2b)^2  \\
D &=& 4ac-4b^2 \\
D &=& 4(ac-b^2)
\end{eqnarray}
Thus, if $ac-b^2 > 0 $ then $D > 0$.
\subsection*{Subassignment 2.1.e}
All results in this subassignment has been solved using Maple. The maple input/output can be found at the end of this document.
\begin{equation}
a=56260, \quad b=730, \quad c=10, \quad d=13761.6, \quad e=192.9 
\end{equation}
\subsection*{Subassignment 2.1.f}
All results in this subassignment has been solved using Maple. The maple input/output can be found at the end of this document.\\\\
I had Maple calculate $M^{-1}$, and could then calculate
\begin{equation}
 \underline{\underline{M}}^{-1} \underline{v} = \begin{pmatrix} -0.1078 \\ 27.15 \end{pmatrix}
\end{equation}
This implies that the straight is defined as
\begin{equation}
y=-0.1078x+27.1578
\end{equation}
The plot of the straight line and the data points can be found in the maple output, at the end of this document.\\ I don't have any comments about the plot, other then it looks right.
\newpage
\section*{Assignment 2.2a}
\subsection*{Subassignment 2.2a.a}
I used Maple for the Gaussian Elimination. After elimination the matrix looked like this
\begin{equation} \underset{=}{A} =
	\begin{pmatrix}
		1 & 0 & -1 \\
		0 & 1 & 3 \\
		0 & 0 & 0
	\end{pmatrix}
\end{equation}
$f$ is bijective when then equation $\underset{=}{M} \underset{=}{X} = \underset{=}{B}$ has one and only one solution for every $\underset{=}{B}$.The book page 51 section 2.4.\\
Theorem 2.3.12(c) says that is $m=n=d$, then the equation system $\underset{=}{M} \underset{=}{X} = \underset{=}{B}$, has only and only one solution for every $\underset{=}{B}$.This does not hold for our equation system where $n>d$ and thus $f$ is not bijective.
\subsection*{Subassignment 2.2a.b}
I added $b_1$, $b_2$ and $b_3$ to the original equation system and used maple to do a Gaussian Elimination.  After elimination the matrix looked like this
\begin{equation} \underset{=}{C}=
	\begin{pmatrix}
		1 & 0 & -1 &  b_1 \\
		0 & 1 & 3 &   b_2-b_1\\
		0 & 0 & 0 &  b_3-b_2+b_1
	\end{pmatrix}
\end{equation}
If the equation system has to be solvable - and thus have at least one solution -  it is easy to see that $b_2-b_3$ has to equal $b_1$.
\end{document}