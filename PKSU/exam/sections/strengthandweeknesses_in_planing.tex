\section{Strength and weekness's in the planing}
I find that we as an developemnt group have had a reasonable allocation of the different
parts in the developmentprocess. In the group we have been good at continuous evaluation 
at our weekly monday meetings and distribute responsibily are mention en the book s. 113. 
This has been very importent because we have been able to 
react fast and adjust the schedule when new and better implementation ideas have come to 
the table og unforeseen changes has occured.

Subsea7 has been a good and reliable business partner in this project, but the planing 
regarding the webservice they should deliver has not been optimal. 
They are very busy and it would have been good for the project, if they  earlier in the process to know in more detail how they would to serve updates to the ipads. In absence we made the system way to advanced, so it would be able to handle incemental updates. If we from the start had known, that they did not intend to serve incemental updates, we would have had a better idea how to analyse and design the application.

In the allocation of resources I also find, that we have had a far too linear approch to analyse, design and implementation, instead of the approach described in M.A.D. experience, Christensen et al. 1998 s. 23
(see figure 7). When we finised the first revision of the analyse, design og implemtation we thought of
it as finale. And later in the process we realised that many revisions were needed to come up with a nice product. So it would have been better to do analyse, design and implemtation more losely, and then repeat the process instead of trying to get it perfect the first time.

Also we have underestimated how long it would take to write proper documentation.

