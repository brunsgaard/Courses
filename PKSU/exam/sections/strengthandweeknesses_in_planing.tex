\section{Strength and weakness's in the planning}
    I find that we as an development team have had a reasonable allocation of
    the different parts in the development process. In the group we have been
    good at continuous evaluation, at our weekly monday meetings and also we
    distributed responsibility as it is mentioned in \cite{oose} s. 113.
    This has been very important because we have been able to react and adjust
    the schedule when new and better implementation ideas came to the
    table or unforeseen changes occurred.

    Subsea7 has been a good and reliable business partner in this project, but
    the planning regarding the webservice they should deliver has not been
    optimal. They are very busy and it would have been good for the project, if
    they earlier in the process would have known how they would like to  serve
    updates to the ipads. Because we did not know how they would serve updates
    we made the system too advanced, so it would be able to handle
    incremental updates. If we had known from the start, that they did not
    intend to serve incremental updates, we would have had a better idea how to
    analyse and correctly design the application earlier in the process.

    In the allocation of resources I also find, that we have had a far too
    linear approach to analyse, design and implementation, instead of the
    approach described in \cite{mad-article} s. 23 (see figure 7). When we
    finished the first revision of the analysis, design and implementation we
    thought of it as more or less final. And only later in the process we
    realised that many revisions were needed to come up with an optimal
    solution. So it would have been better to do a more rough analysis, design and
    implementation, and then repeat and refine the process instead of trying to
    get it perfect the first time.
