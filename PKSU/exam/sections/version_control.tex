\section{Version control}
\label{sec:version_control}
After meeting with Subsea7 in Stavanger on the 23rd and 24th of May,
quite a lot of the code base has changed. Originally we've programmed
with a philisophy of being able to handle an arbitrary number of
employees, department and projects. For the application to work in a
satisfying manner with dealing of thousands of objects, we opted for
supporting partial updates allowing to update a specific object (be it
an employee, a department or a project) without sending a full
record. This would allow us to reduce the amount of data being sent on
the network by an order of magnitude or more.

While this is a great feature, and possibly essential in some
settings, the last meeting with Subsea7 clarified their needs and
infrastructure, which has allowed us to substantially reduce the code
base by removing some of the functionality.

Comparing the last revision before the meeting with Subsea7 with the
latest revision, the former consists of 3,558 standard lines of code (sloc)
while the latter contains 3,024 sloc.\footnote{Both figures have been generated using
  David A. Wheeler's 'SLOCCount'.}  This corresponds to a reduction of
sloc in the order of ~15\%. Taken into account that the latter
revision further adds some functionality not considered before the
meeting (e.g. local scaling of images), an estimate of the real
reduction in sloc may be as high as 20\%.

Some of the most drastic changes in the code base are as follows:

\begin{description}

    \item[Config] \hfill \\
        Configuration functionality has been revamped as to avoid a cluttered
        structure. The \texttt{Config} subsystem now exists in a single file and has
        support for default value for configuration settings.

    \item[DBWrapper] \hfill \\
        After the meeting with Subsea7, we decided to abandon the database
        back-end of the application, as it has been made redundant with the
        decision to drop support for partial updates.

    \item[Model] \hfill \\
        Due to the \texttt{DBWrapper} being removed, the \texttt{Model} could
        be simplified significantly. This has lead to a much more maintainable
        codebase.

\end{description}
