\section{Implementation of the parser}
    This section is about the decision making and implementation of the parser.
    I have chosen the parser because it has been rewritten in the process and I
    will explain why this was a good idea.

    First, take a look at the initial parser code parsing the employees from
    the xml file

    \begin{lstlisting}
XNamespace f = "http://internal.s7.net/S7finder";
XDocument.Load (dataFeed);

XElement employeeRoot = doc.Root.Element (f + "employees"); 
if (employeeRoot !=null) { 
    
IEnumerable<XElement> employees = employeeRoot.Elements (f + "employee");

foreach (XElement e in employees) {

    if (e.Attribute ("delete") != null && 
                Convert.ToBoolean (e.Attribute("delete").Value)) 
    { 
    model.Employee.delete (db, Convert.ToInt64(e.Attribute ("id").Value));
    } else { 
        byte[] image = Convert.ToByte64(e.Element (f +"image").Value); 
        Int64 employeeID = Convert.ToInt64(e.Attribute ("id").Value); 
        string firstName = e.Element (f + "firstName").Value; string 
        lastName = e.Element (f +"lastName").Value; 
        string title = e.Element (f + "title").Value;
        Int64 phone = Convert.ToInt64 (e.Element (f + "phone").Value);
        string location = e.Element (f + "location").Value;

        model.Employee.create ( db,
                                employeeID, 
                                firstName, 
                                lastName, 
                                title, 
                                phone, 
                                location, 
                                image, 
                                thumbnail) 
        } 
    } 
}
    \end{lstlisting}

    This code was part of the very first parser. As the reader can see from the
    code, the parser looks for an atirbute called delete. If the the delete
    attribute is true, then we delete the employee from the db, else we insert.
    Also at this stage in the analysing we have not yet figured out, that we would
    fetch employee pictures from a webserver, instead of part as the xml feed. Also
    after more extensive research in the field of C\# and XML, we found the LINQ
    package, could help us with readability and robustness. So we choose to rewrite
    the code. Below are shown the same code snip after the rewrite:

    \begin{lstlisting}
var ns = xdoc.Root.Name.Namespace; var results = from el in xdoc.Root.Elements
(ns + "employees").Elements (ns + "employee") select el;

foreach (var r in results) {

    ulong id = (ulong)r.Attribute ("id"); string firstName = (string)r.Element
    (ns + "firstName"); string lastName = (string)r.Element (ns + "lastName");
    string title = (string)r.Element (ns + "title"); string phone =
    (string)r.Element (ns + "phone"); string location = (string)r.Element (ns +
    "location"); Status status = (Status)Enum.Parse ( typeof(Status), r.Element
    (ns + "status").Value); ulong department = (ulong)r.Element (ns +
    "department");

EmployeeHandler.Insert (id, firstName, lastName, title, phone, location,
status, department );
    \end{lstlisting}
    After we figured these things out and reimplemented the parser, we had the
    following improvements.








