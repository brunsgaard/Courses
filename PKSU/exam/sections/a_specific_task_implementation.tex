\section{Implementation of the parser}
    This section is about the decision making and implementation of the parser.
    I have chosen the parser because it has been rewritten in the process and I
    will explain why this was a good decision.

    First, take a look at the initial parser code, parsing the employees from
    the XML file.
    \clearpage
    \begin{lstlisting}
XNamespace f = "http://internal.s7.net/S7finder";
XDocument doc = XDocument.Load (dataFeed);

XElement employeeRoot = doc.Root.Element (f + "employees");
if (employeeRoot != null) {
    IEnumerable<XElement> employees = 
        employeeRoot.Elements (f + "employee");

    foreach (XElement e in employees) {

        if (e.Attribute ("delete") != null && 
            Convert.ToBoolean (e.Attribute ("delete").Value)) {
            model.Employee.delete (db,Convert.ToInt64(e.Attribute("id").Value)); 
        } else {
            byte[] test = {1,2,3,4};
            Int64 employeeID = Convert.ToInt64 (e.Attribute ("id").Value);
            string firstName = e.Element (f + "firstName").Value;
            string lastName = e.Element (f + "lastName").Value;
            string title = e.Element (f + "title").Value;
            Int64 phone = Convert.ToInt64 (e.Element (f + "phone").Value);
            string location = e.Element (f + "location").Value;

            model.Employee.create (db, employeeID, firstName, lastName, 
                                            title, phone, location, test, test);    
        }
}}
    \end{lstlisting}
    The code above was part of the very first parser. 
    
    This simple part of the parser looks for an attribute called delete. If the
    delete attribute is true, then we delete the employee from the database,
    else we insert the employee.

    After more research in the field of C\# and XML, we found the LINQ package,
    could help us with readability and robustness. Also we had some changes to
    the design of the application, as an example the images and thumbnails
    should no longer be a part of the XML file, but was loaded separately from
    a webserver, based on the employeeId.

    So we had more reasons to do a code rewrite. 
    Below are shown the same code snip after the rewrite:
    \begin{lstlisting}
var ns = xdoc.Root.Name.Namespace; var results = from el in xdoc.Root.Elements
(ns + "employees").Elements (ns + "employee") select el;

foreach (var r in results) {

    ulong id = (ulong)r.Attribute ("id"); 
    string firstName = (string)r.Element
    (ns + "firstName");
    string lastName = (string)r.Element (ns + "lastName");
    string title = (string)r.Element (ns + "title"); 
    string phone = (string)r.Element (ns + "phone"); 
    string location = (string)r.Element (ns + "location"); 
    Status status = (Status)Enum.Parse ( typeof(Status), 
                                         r.Element (ns + "status").Value);
    ulong department = (ulong)r.Element (ns + "department");

EmployeeHandler.Insert (id, firstName, lastName, title, phone, location,
status, department );
    \end{lstlisting}
    After the code rewrite I think that the code was improved in the following
    ways. In regard to readability, the code now uses LINQ, thus a SQL-like
    syntax "from * in * select" are used, this syntax should be well known for
    many programmers, so in that regard I will argue that we improved
    readability.
    
    We found that LINQ, did a better job at handling flawed XML files and
    dynamic namespaces, thus we improved robustness.
    
    Because we found a more optimal solution for image fetching, the parser
    has become more simple. I think in all way this code rewrite improved the
    implementation of the parser.
