\section{Literature review}
In the following section we will review two articles, namely
``Programming as theory building'' by Peter Naur (of DIKU fame), and
``The M.A.D. Experience: Multiperspective Application Development in
evolutionary prototyping'' by Michael Christensen, Andy Crabtree,
Christian Heide Damm, Klaus Marius Hansen, Ole Lehrmann Madsen,
Pernille Marqvardsen, Preben Mogensen, Elmer Sandvad, Lennert Sloth,
Michael Thomsen from the computer science department at Århus
University. \cite{naur, mad-article}

\subsection{Programming as theory building}
In ``Programming as theory building'', Peter Naur discusses the notion
of programming not as a discipline where program code is produced
(``program texts'' in the article), but rather a discipline where the
product is the design, including documentation and rationale for the
design decisions.

To support the Theory Building View, Naur lists a few anecdotes, which
we may gain some insight from, and draw some parallels
to.

With our project, we're essentially working as external software
developers, catering to the needs of a client. Some of the
restrictions regarding implementation are based on the hardware
platform, while others are requirements set by Subsea7. One of these
restrictions is that the code should be developed in C\#, in order to
enable developers at Subsea7 to continue development if necessary.

In \textit{Program Life, Death and Revival}, Naur argues strongly that
it is insufficient for a new programmer joining a project to become familiar
with a certain program and its corresponding documentation, but that
it is an actual requirement to work alongside other programmers with
an intimate knowledge of the system, as to convey the program
theory. By extension, a revival of a program when following the Theory
Building views, is impossible, as the need for a means of conveying
the program theory remains unsatisfied.

As external developers, the project we're delivering has some
detachment issues when considering Naurs model, as at is unlikely that
a future developer would take contact to our project group. One might
argue that the need for contact follows the size of the project, and
that for a project of the size of S7Finder, any developer should be
able to understand the code after reading it a few times. On the other
hand, we may also argue that some of the original design ideas are
likely to be lost, and inconsistency may arise. Although the code is
fairly well-documented, some features may seem less than intuitive,
e.g. the use of file timestamps and the \texttt{touch}-ing of
files.\footnote{\texttt{touch} is a standard Unix/Linux command, that in its
  base case updates the timestamp of a file to ``now''.}
This may can be seen as a loss of theory, hence supporting Naurs reasoning.

Though the text is written in 1985, 27 years ago, the notion of
programming as theory building as opposed to text production still
holds. Drawing parallels from our own project, we realised during the
fourth meeting with Subsea7 at their offices in Stavanger, as
discussed in \aref{sec:version_control}, that they wouldn't make use
of the originally implemented features.

\subsection{The M.A.D. Experience}

The article ``The M.A.D. Experience: Multiperspective Application
Development in evolutionary prototyping'', the development of an
application, a Global Customer Service System (GCSS) for a Danish
goods transport company is described. Although the article is
extremely explicit and favours unreasonably long paragraphs (a third
of a page is not uncommon) and intellectual-sounding words over
simplicity, while bordering on incomprehensibility on account of the
required level of understanding of ethnography, some valuable points
may be found in the article.\footnote{E.g.: what is the difference between a
  ``naturalistic description'' and a regular one?  (pg. 10)}

As we have demonstrated in our development process, the design phase is a
cooperative activity, and at our visits to Stavanger, we took the
opportinuty to act on inputs from Subsea7. In the development of the
GCSS, the group from CIT followed a similar model, although much more
in-depth, as the GCSS project is bigger both in scope and
duration. Where we thrice took the trip to Stavanger and listened to
ideas, complaints and constructive criticism, the CIT group had
opportunity to visit several offices, both locally and globally
(Malaysia, Singapore), for a duration of multiple weeks.

In some places the two design processes differ.
In ``The object-oriented perspective'', the CIT group notes that their
focus was ``...on modelling as opposed to technical concepts like
encapsulation and inheritance''. As this has been the first major OO
project done by either of the group members, our modelling have served
the same goals, but might have been more based on programming
experience rather than pure modelling. Our choice of programming
language and development platforms may also have been a contributer to
the difference in modelling, as we've had the task to not only
structure the project, but also learn a new programming language (only
one group member had any real experience with C\#) and a full set of
development tools (XCode and MonoDevelop). Further, our group consists
of four programmers, while the CIT group consisted of an ethnographer,
a participatory designer and developers.

As they experienced, noted in ``The evolution of the model'', our
initial model differs quite a bit from the final model. As explained
in \aref{sec:version_control}, the final meeting with Subsea7 resulted
in removal of some features, and the simplification of others. While
it may have eased the design process to have an initial model more
like the result, it is our experience that this hasn't been a
hinderance. Rather like the conclusions of the CIT group, it is our
understanding that ``a model cannot be effectively created in
isolation from implementation''. While this may not always be true,
preserving true separation between code and model might be detrimental
to implementation efforts.
