\documentclass[11pt,a4paper]{article}

\usepackage[utf8]{inputenc}
\usepackage[english]{babel}
\usepackage[T1]{fontenc}

\usepackage{amsmath,amssymb,amsfonts}

\title{Test Documentation}
\author{S7 Dev team}

\begin{document}
\maketitle

\part{Test case specification} % (fold)
\label{prt:Test_case_specification}

Indentifier: ModelHandlerTests

\section{Test items} % (fold)
\label{sec:Test_items}

\fxfatal{Insert test case figure here}

This tests case covers the handlers for the model objects. These handlers
control the life cycle of all model objects, and are responsible for
caching them. All features of the handlers will be tested to ensure
expected behaviour. This includes the throwing of exceptions in error cases.
Random input is fed to the classes to ensure that all edge cases are found.

% section Test items (end)

\section{Input specifications} % (fold)
\label{sec:Input_specifications}

To ensure robust testing, all inputs are randomly generated. This includes
IDs for non-existant employees, and the values for randomly generated
employee, department and project objects. This enables us to think about
the conceptual operations the methods under testing perform, instead of
the specifics of one object.

% section Input specifications (end)

\section{Output specifications} % (fold)
\label{sec:Output_specifications}

Outputs from all methods are tested against the randomly generated values.
No transformations are made on the data, as all processing is done on the
gateway server. This enables simple output tests which contain logic that
is easy to validate.

% section Output specifications (end)

\section{Environmental needs} % (fold)
\label{sec:Environmental_needs}

\fxfatal{Reference}
As specified in the Test Plan, all tests must be run in the iOS Simulator through the
Monotouch port of NUnit.

% section Environmental needs (end)

% part Test case specification (end)

\end{document}
